\subsection*{外観}



\subsection*{ピンアサイン}

\tabulinesep=1mm
\begin{longtabu}spread 0pt [c]{*{2}{|X[-1]}|}
\hline
\PBS\centering \cellcolor{\tableheadbgcolor}\textbf{ pin番号  }&\PBS\centering \cellcolor{\tableheadbgcolor}\textbf{ 接続先   }\\\cline{1-2}
\endfirsthead
\hline
\endfoot
\hline
\PBS\centering \cellcolor{\tableheadbgcolor}\textbf{ pin番号  }&\PBS\centering \cellcolor{\tableheadbgcolor}\textbf{ 接続先   }\\\cline{1-2}
\endhead
D2  &ゴール前スイッチ入力   \\\cline{1-2}
D3  &サーボ   \\\cline{1-2}
D4  &D-\/subゴール通知ピン   \\\cline{1-2}
D5  &D-\/subコース接触通知ピン   \\\cline{1-2}
D7  &ゴール上\+L\+ED   \\\cline{1-2}
D8  &ゴール下\+L\+ED   \\\cline{1-2}
D9  &コース電圧入力   \\\cline{1-2}
D10$\sim$13  &D\+I\+Pロータリースイッチ   \\\cline{1-2}
D17  &フォトインタラプタ入力   \\\cline{1-2}
\end{longtabu}


\subsection*{iraira\+\_\+goal\+\_\+main.\+ino}


\begin{DoxyItemize}
\item メインプログラム
\item loop()内の処理は以下の2つの処理を行うだけのシンプルなつくり
\begin{DoxyItemize}
\item イベント発生確認
\begin{DoxyItemize}
\item 以下の3つのイベントが発生しているか順に確認し、発生していたら該当するイベント番号を返す
\begin{DoxyItemize}
\item ゴール前スイッチが押されているか
\item コースに接触しているか
\item フォトインタラプタ通過中か
\end{DoxyItemize}
\end{DoxyItemize}
\item イベント対応処理実行
\begin{DoxyItemize}
\item 引数であるイベント番号に対応した処理を実行する
\item イベントが発生していない場合は何もしない
\end{DoxyItemize}
\end{DoxyItemize}
\item Standard\+Cplusplus.\+h という外部ライブラリを使用している
\begin{DoxyItemize}
\item このライブラリを使うことでc++の\+S\+T\+Lを使用できる
\end{DoxyItemize}
\end{DoxyItemize}

\subsection*{\mbox{\hyperlink{led__manager_8hpp}{led\+\_\+manager.\+hpp}} $\vert$ cpp}


\begin{DoxyItemize}
\item ゴール上下にある\+L\+E\+D制御用クラス
\item 2つの\+L\+E\+Dをまとめて操作する
\item インスタンス生成時に\+L\+E\+Dのピン番号をリストで指定する
\item 現在は\+L\+E\+Dを2つ制御しているが任意個の\+L\+E\+Dを制御可能
\item 点滅処理はdelay()で時間制御をしているため、一定時間ほかの処理を行いないことに注意
\item iraira\+\_\+goal\+\_\+main.\+inoと同様に\+Standard\+Cplusplus.\+hをインクルードしている 
\end{DoxyItemize}